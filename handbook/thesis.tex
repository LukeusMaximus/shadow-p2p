% The document class marks this as a thesis, supplying various options that
% control rendering of some standard features (e.g., the cover page).

\documentclass[ % the name of the author
                    author={Daniel Page},
                % the name of the supervisor (preferably including title)
                supervisor={Dr. Ian Holyer},
                % the thesis    title (which cannot be blank)
                     title={Some Structural Guidelines for CS MEng Theses},
                % the thesis subtitle (which can    be blank)
                  subtitle={},
                % the degree programme (from BSc, MEng, MSci, MSc and PhD)
                    degree={MEng},
                % the year of submission
                      year={2012} ]{thesis}

\begin{document}

% =============================================================================

% This section simply introduces the structural guidelines.  It can clearly
% be deleted (or commented out) if you use the file as a template for your
% own thesis: everything following it is in the correct order to use as is.

\section*{Prelude}
\thispagestyle{empty}

A typical thesis will be structured according to a number of standard 
sections described in what follows.  However, it is hard and perhaps
even counter-productive to generalise: the goal of outlining this 
typical structure is {\em not} to be prescriptive, but simply to act 
as a guideline.  In particular, the page counts given are important 
but not absolute: their aim is simply to highlight that a clear, 
concise description is better than a rambling alternative that 
makes it hard to separate important content and facts from trivia.

You can use this document as a \LaTeX-based~\cite{latexbook1,latexbook2} 
template for your own thesis by simply deleting extraneous sections (e.g., 
this one); keep in mind that the associated {\tt Makefile} could be of
use, in particular since it also executes \mbox{\BibTeX} to deal with the
bibliography.  If you opt not to do this, which is perfectly acceptible,
a standard cover and declaration of authorship produced online via
\[
\mbox{\url{http://www.cs.bris.ac.uk/Teaching/Resources/cover.html}}
\]

% =============================================================================

% This macro creates the standard UoB title page, with information drawn
% from the document class (meaning it is vital you select the correct
% degree title and so on).

\maketitle

% After the title page (which is a special case in that it is not numbered)
% comes the front matter or preliminaries; this macro signals the start of
% such content, meaning the pages are numbered with Roman numerals.

\frontmatter

% This macro creates the standard UoB thesis declaration; on the hard-copy,
% this must be signed by the author in the space indicated.

\makedecl

% LaTeX will automatically generate a table of contents, and also associated 
% lists of figures, tables and algorithms.  The former is a compulsory part
% of the thesis, but if you do not require the latter they can be suppressed
% by simply commenting out the associated macro.

\tableofcontents
\listoffigures
\listoftables
\listofalgorithms
\lstlistoflistings

% The following sections are part of the front matter, but are not generated
% automatically by LaTeX; the use of \chapter* means they are not numbered.

% -----------------------------------------------------------------------------

\chapter*{Executive Summary}

{\bf A compulsory section, of at most $1$ page} 
\vspace{1cm} 

\noindent
This section should pr\'{e}cis the project context, aims and objectives 
and main contributions and achievements; the same section may be called
an abstract elsewhere.  The goal is to ensure the reader is clear about 
what the topic is, what you have done within this topic, {\em and} what 
your view of the outcome is.

The former aspects should be guided by your specification: essentially 
this section is a (very) short version of what is typically the first 
chapter.  The latter aspects should be presented as a concise, factual 
bullet point list that summarises the main contributions and achievements.  
The points will of course differ for each project, but an example is as 
follows:

\begin{quote}
\noindent
\begin{itemize}
\item I spent $120$ hours collecting material on and learning about the 
      Java garbage-collection sub-system. 
\item I wrote a total of $5000$ lines of source code, comprising a Linux 
      device driver for a robot (in C) and a GUI (in Java) that is 
      used to control it.
\item I designed a new algorithm for computing the non-linear mapping 
      from A-space to B-space using a genetic algorithm, see page $17$.
\item I implemented a version of the algorithm proposed by Jones and 
      Smith in [6], see page $12$, corrected a mistake in it, and 
      compared the results with several alternatives.
\end{itemize}
\end{quote}

% -----------------------------------------------------------------------------

\chapter*{Supporting Technologies}

{\bf A compulsory section, of at most $1$ page}
\vspace{1cm} 

\noindent
This section should present a detailed summary, in bullet point form, of 
any third-party resources (e.g., hardware and software components) used 
during the project.  Use of such resources is perfectly acceptable: the 
goal of this section is simply to be clear about where and how they are 
used.  The content can focus on the project topic itself (rather than, 
for example, including ``I used \mbox{\LaTeX} to prepare my thesis''); 
an example is as follows:

\begin{quote}
\noindent
\begin{itemize}
\item I used the Java {\tt BigInteger} class to support my implementation 
      of RSA.
\item I used a parts of the OpenCV computer vision library to capture 
      images from a camera, and for various standard operations (e.g., 
      threshold, edge detection).
\item I used an FPGA device supplied by the Department, and altered it 
      to support an open-source UART core obtained from 
      \url{http://opencores.org/}.
\item The web-interface component of my system was implemented by 
      extending the open-source WordPress software available from
      \url{http://wordpress.org/}.
\end{itemize}
\end{quote}

% -----------------------------------------------------------------------------

\chapter*{Notation and Acronyms}

{\bf An optional section, of roughly $1$ or $2$ pages}
\vspace{1cm} 

\noindent
Any well written document will introduce notation and acronyms before their 
use, {\em even if} they are standard in some way: this ensures any reader 
can understand the resulting self-contained content.  

Said introduction can exist within the thesis itself, wherever that is
appropriate.  For an acronym, this is typically achieved at the first point 
of use via ``Advanced Encryption Standard (AES)'' or similar, noting the 
capitalisation of relevant letters.  However, it can be useful to include 
an additional, dedicated list at the start of the thesis; the advantage of 
doing so is that you cannot mistakenly use an acronym, for example, 
before defining it.  An example is as follows:

\begin{quote}
\noindent
\begin{tabular}{lcl}
AES                 &:     & Advanced Encryption Standard            \\
DES                 &:     & Data Encryption Standard                \\
                    &\vdots&                                         \\
${\mathcal H}( x )$ &:     & the Hamming weight of $x$               \\
${\mathbb  F}_q$    &:     & a finite field with $q$ elements        \\
$x_i$               &:     & the $i$-th bit of some bit-sequence $x$ \\
\end{tabular}
\end{quote}

% -----------------------------------------------------------------------------

\chapter*{Acknowledgements}

{\bf An optional section, of at most $1$ page}
\vspace{1cm} 

\noindent
It is common practice (although totally optional) to acknowledge any
third-party advice, contribution or influence you have found useful
during your work.  Examples include support from friends or family, 
the input of your Supervisor, external organisations or persons who 
have supplied resources of some kind (e.g., funding, advice or time),
and so on.

% =============================================================================

% After the front matter comes a number of chapters; under each chapter,
% sections, subsections and even subsubsections are permissible.  The
% pages in this part are numbered with Arabic numerals.  Note that:
%
% - A reference point can be marked using \label{XXX}, and then later
%   referred to via \ref{XXX}; for example Chapter\ref{chap:context}.
% - The chapters are presented here in one file; this can become hard
%   to manage.  An alternative is to save the content in seprate files
%   the use \input{XXX} to import it, which acts like the #include
%   directive in C.

\mainmatter

% -----------------------------------------------------------------------------

\chapter{Contextual Background}
\label{chap:context}

{\bf A compulsory chapter, of roughly $10$ pages}
\vspace{1cm} 

\noindent
This chapter is intended to describe the project context, and motivate
the proposed aims and objectives.  Ideally, it is written at a fairly 
high-level, and easily understood by a reader who is technically 
competent but not an expert in the topic itself.

In short, the goal is to answer three questions for the reader.  First, 
what is the project topic, or problem being investigated?  Second, why 
is the topic important, or rather why should the reader care about it?  
For example, why there is a need for this project (e.g., lack of similar 
software or deficiency in existing software), who will benefit from the 
project and in what way (e.g., end-users or software developers, or 
researchers), what work does the project build on and why is the selected 
approach important or interesting (e.g., fills a gap in literature, applies
results from another field to a new problem).  Finally, what are the 
central challenges involved and why are they significant? 
 
The chapter should conclude with a concise bullet point list that 
summarises the aims and objectives.  For example:

\begin{quote}
\noindent
The high-level objective of this project is to reduce the performance 
gap between hardware and software implementations of modular arithmetic.  
More specifically, the concrete aims are:

\begin{enumerate}
\item Research and survey literature on public-key cryptography and
      identify the state of the art in exponentiation algorithms.
\item Improve the state of the art algorithm so that it can be used
      in an effective and flexible way on constrained devices.
\item Implement a framework for describing exponentiation algorithms
      and populate it with suitable examples from the literature on 
      an ARM7 platform.
\item Use the framework to perform a study of algorithm performance
      in terms of time and space, and show the proposed improvements
      are worthwhile.
\end{enumerate}
\end{quote}

% -----------------------------------------------------------------------------

\chapter{Technical Background}
\label{chap:technical}

{\bf A compulsory chapter, of roughly $10$ to $20$ pages} 
\vspace{1cm} 

\noindent
This chapter is intended to describe the technical basis on which execution
of the project depends.  The goal is to provide a detailed explanation of
the specific problem at hand, and any previous or related work in the area 
(e.g., descriptions of supporting technologies, existing algorithms that 
you use, alternative solutions proposed).  

Put another way, after reading this chapter a non-expert reader should have 
obtained enough background to understand what {\em you} have done, and then
assess how novel, challenging and rigorous your work is.  You might view an 
additional goal as giving the reader confidence that you are able to absorb 
and understand research-level material.

% -----------------------------------------------------------------------------

\chapter{Project Execution}
\label{chap:execution}

{\bf A topic-specific chapter, of roughly $20$ pages} 
\vspace{1cm} 

\noindent
This chapter is intended to describe what you did: the goal is to explain
the main activity or activities, of any type, which constituted your work 
during the project.  The content is highly topic-specific, but for many 
projects it will make sense to split the chapter into two sections: one 
will discuss the design of something (e.g., some hardware or an algorithm), 
inc. any rationale or decisions made, and the other will discuss how this 
design was realised via some form of implementation.  

This is, of course, far from ideal for {\em many} project topics.  Some
situations which clearly require a different approach include:

\begin{itemize}
\item In a project where asymptotic analysis of some algorithm is the goal,
      there is no real ``design and implementation'' in a traditional sense
      even though the activity of analysis is clearly within the remit of
      this chapter.
\item In a project where analysis of some results is as major, or a more
      major goal than the implementation that produced them, it might be
      sensible to merge this chapter with the next one: the main activity 
      is such that discussion of the results cannot be viewed separately.
\end{itemize}

\noindent
Note that evidence of ``best practice'' project management (e.g., use of 
version control, choice of programming language and  so on) should only 
be included if there is a clear reason to do so.

\section{Example Section}

This is an example section; 
the following content is auto-generated dummy text.
\lipsum

\subsection{Example Sub-section}

\begin{figure}[t]
\centering
foo
\caption{This is an example figure.}
\label{fig}
\end{figure}

\begin{table}[t]
\centering
\begin{tabular}{|cc|c|}
\hline
foo      & bar      & baz      \\
\hline
$0     $ & $0     $ & $0     $ \\
$1     $ & $1     $ & $1     $ \\
$\vdots$ & $\vdots$ & $\vdots$ \\
$9     $ & $9     $ & $9     $ \\
\hline
\end{tabular}
\caption{This is an example table.}
\label{tab}
\end{table}

\begin{algorithm}[t]
\For{$i=0$ {\bf upto} $n$}{
  $t_i \leftarrow 0$\;
}
\caption{This is an example algorithm.}
\label{alg}
\end{algorithm}

\begin{lstlisting}[float={t},caption={This is an example listing.},label={lst},language=C]
for( i = 0; i < n; i++ ) {
  t[ i ] = 0;
}
\end{lstlisting}

This is an example sub-section;
the following content is auto-generated dummy text.
Notice the examples in Figure~\ref{fig}, Table~\ref{tab}, Algorithm~\ref{alg}
and Listing~\ref{lst}.
\lipsum

\subsubsection{Example Sub-sub-section}

This is an example sub-sub-section;
the following content is auto-generated dummy text.
\lipsum

\paragraph{Example paragraph.}

This is an example paragraph; note the trailing full-stop in the title,
which is common style intended to ensure it does not run into the text.

% -----------------------------------------------------------------------------

\chapter{Critical Evaluation}
\label{chap:evaluation}

{\bf A topic-specific chapter, of roughly $10$ pages} 
\vspace{1cm} 

\noindent
This chapter is intended to evaluate what you did.  The content is highly 
topic-specific, but for many projects will have flavours of the following:

\begin{enumerate}
\item functional testing, inc. analysis of failure cases,
\item performance results, and analysis of said results that draw some 
      form of conclusion,
      and
\item evaluation of options and decisions within the project, and/or a
      comparison with alternatives.
\end{enumerate}

\noindent
This chapter often acts to differentiate project quality: even if the work
completed is of a high technical quality, critical yet objective evaluation 
and comparison of the outcomes is crucial.  In essence, the reader wants to
learn something, so the worst examples amount to simple statements of fact 
(e.g., ``graph X shows the result is Y''); the best examples are analytical 
and exploratory (e.g., ``graph X shows the result is Y, which means Z; this 
contradicts [1], which may be because I use a different assumption'').  As 
such, both positive {\em and} negative outcomes are valid {\em if} presented 
in a suitable manner.

% -----------------------------------------------------------------------------

\chapter{Conclusion}
\label{chap:conclusion}

{\bf A compulsory chapter, of roughly $2$ pages} 
\vspace{1cm} 

\noindent
The concluding chapter of a thesis is often underutilised, in part because
it is often left until close to the deadline and hence does not get enough 
attention.  Ideally, the chapter will consist of three parts:

\begin{enumerate}
\item (Re)summarise the main contributions and achievements, in essence
      summing up the content.
\item Clearly state the current project status (e.g., ``X is working, Y 
      is not'') and evaluate what has been achieved with respect to the 
      initial aims and objectives (e.g., ``I completed aim X outlined 
      previously, the evidence for this is within Chapter Y'').  There 
      is no problem including aims which were not completed, but it is 
      important to evaluate and/or justify why this is the case.
\item Outline any open problems or future plans.  Rather than treat this
      only as an exercise in what you {\em could} have done given more 
      time, try to focus on any unexplored options or interesting outcomes
      (e.g., ``my experiment for X gave counter-intuitive results, this 
      could be because Y and would form an interesting area for further 
      study'').
\end{enumerate}

% =============================================================================

% Finally, after the main matter, the back matter is specified.  This is
% typically populated with just the bibliography.  LaTeX deals with these
% in one of two ways, namely
%
% - inline, which roughly means the author specifies entries using the 
%   \bibitem macro and typesets them manually, or
% - using BiBTeX, which means entries are contained in a separate file
%   (which is essentially a databased) then inported; this is the 
%   approach used below, with the databased being thesis.bib.
%
% Either way, the each entry has a key (or identifier) which can be used
% in the main matter to cite it, e.g., \cite{X}, \cite[Chapter 2}{Y}.

\backmatter

\bibliography{thesis}

% -----------------------------------------------------------------------------

% The thesis concludes with a set of (optional) appendicies; these are the
% same as chapters in a sense, but once signaled as being appendicies via
% the associated macro, LaTeX manages them appropriatly.

\appendix

\chapter{An Example Appendix}
\label{appx:example}

Content which is not central to, but may enhance the thesis can be
included in one or more appendices; examples include, but are not 
limited to

\begin{itemize}
\item lengthy mathematical proofs, numerical or graphical results
      which are summarised in the main body,
\item sample or example calculations, 
      and
\item results of user studies or questionnaires.
\end{itemize}

\noindent
Note that in line with most research conferences, the marking panel 
is not obliged to read such appendices.

% =============================================================================

\end{document}
